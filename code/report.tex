\documentclass[letter,11pt]{article}
\usepackage[utf8x]{inputenc}
\usepackage[top=1in, bottom=1in, left=1in, right=1in]{geometry}
\usepackage{url}

%opening
\title{Using NLP Techniques for File Fragment Classification}
\author{Simran Fitzgerald, George Mathews, Colin Morris and Oles Zhulyn}

\begin{document}

\maketitle

\begin{abstract}
The classification of file fragments is an important problem in digital forensics. The literature does not include comprehensive work on applying machine learning techniques to this problem. In this work, we explore the use of techniques from natural language processing to classify file fragments. We took a supervised learning approach, based on the use of support vector machines combined with the bag-of-words model, where text documents are represented as unordered bags of words. This technique has been repeatedly shown to be effective and robust in classifying text documents (e.g., in distinguishing positive movie reviews from negative ones).

In our approach, we represent file fragments as ``bags of bytes'' with feature vectors consisting of unigram and bigram counts, as well as other statistical measurements (including entropy and others). We made use of the publicly available Garfinkel data corpus to generate file fragments for training and testing. We ran a series of experiments, and found that this approach is effective and robust in this domain as well.
\end{abstract}

\section{Introduction}
\label{Section:Introduction}
The classification of file fragments is an important problem in digital forensics, particularly for the purpose of carving fragmented files. Files that are not stored contiguously on the hard drive must be carefully reconstructed from fragments based on their content during file carving. Because the search space for fragments belonging to a particular file is so large, it is essential to have an automated method for distinguishing whether a fragment potentially belongs to a file or not. For example, a fragment from a plain-text file (e.g. \texttt{txt}) certainly does not belong to a compressed image file (e.g. \texttt{jpg}). Garfinkel \cite{Garfinkel07} observes that although the fragmentation of files is relatively rare on today's file systems, the files of interest in forensic investigations are more likely to be fragmented than other files. For example, large files that have been modified numerous times over a long period of time on a hard drive that is filled near capacity will likely exhibit fragmentation. In this work, we explore the application of machine learning techniques from natural language processing to the problem of file fragment classification.

Classification is a standard machine learning problem. There has been work done on applying machine learning techniques to the problem of file fragment classification \cite{Axelsson10, Conti10, Li10, Veenman07}. However, the body of work in the literature is not exhaustive. Our contribution is the application of supervised machine learning techniques used in natural language processing to this problem.

In previous work, the histograms of the bytes within file fragments are used for classification \cite{Li05, Li10, Stolfo05, Veenman07}. In natural language processing, this kind of approach is called the ``bag-of-words model'', where text documents are represented as unordered bags of words. The single word tokens are called unigrams, but tokens consisting of any fixed number of words can also be considered. Two word tokens are called bigrams, and bigram counts capture more information about the structure of the data being classified than do unigram counts alone. Combined with various machine learning techniques, this kind of approach has been repeatedly shown to be effective and robust in classifying text documents (e.g. in determining whether a piece of text has a positive or negative sentiment). In our approach, we consider the unigram and bigram counts of the bytes within file fragments, along with other statistical measurements, to generate feature vector representations of the file fragments, which we then classify based on 23 different file types using a support vector machine.

Support vector machines are supervised machine learning algorithms that are very effective for classification problems \cite{Li10}. During the training phase, a support vector machine partitions a high-dimensional space based on the points it contains that belong to known classes. File fragments can be represented in the high-dimensional space by being transformed into feature vectors. During the testing phase, file fragments of unknown types are transformed into feature vectors and are classified according to what partition they lie in in the high-dimensional space. We make use of the \texttt{libsvm} \cite{CC01a} library, which is the one of the most widely used implementations of support vector machines, to perform our experiments.

Most of the previous work on this problem exclusively uses private data sets, making it more difficult for other researchers to reproduce experimental results. We follow the example of Axelsson \cite{Axelsson10} and derive the data set we use for training and testing from the freely available corpus of forensics research data by Garfinkel et al. \cite{Garfinkel09} (the \texttt{govdocs1} data set described in Section 4 of the cited paper). We determined the most well-represented file types in the data set and selected 23 of them based on how well-known they are. For each of the 23 file types supported by our classifier, we downloaded files uniformly at random from the \texttt{govdocs1} data set such that we would have at least 10 files made up of at least 10000 512-byte fragments. From this, we uniformly at random selected 9000 fragments for each file type to create our data set. When generating the fragments, we omitted the first and last fragments of each file, as the first fragment frequently contains header information that identifies the file type, and the last fragment might not be 512 bytes in length.

\textbf{TODO mention experimental setup and hint at results}

The paper is organized as follows. In Section \ref{Section:RelatedWork}, we provide a brief overview of related work done in this area. In Section \ref{Section:ExperimentalSetup}, we describe our experimental setup, which includes how we generated the data set we used in training and testing, as well as details about the features we used in our feature vectors. In Section \ref{Section:Results}, we present our results. We conclude and suggest future directions for this work in Section \ref{Section:ConclusionAndFutureWork}.

\section{Related work}
\label{Section:RelatedWork}
Previous work that explores the application of machine learning techniques to the problem of file fragment classification appears in the literature.

Calhoun and Coles \cite{Calhoun08} considered only four file types (namely, \texttt{jpg}, \texttt{bmp}, \texttt{gif}, and \texttt{pdf}). For each pair of these file types, they used linear discriminant analysis \cite{Fisher36} (which is used to find linear combinations of features to characterize or separate objects between classes) in order to classify a file fragment as having one type or the other. The features they considered were various statistical measurements, including Shannon entropy \cite{Shannon48} and frequency of ASCII codes. They achieved fairly good accuracy (88.3\%). However, since they only classified fragments based on only four file types on a pairwise basis, it is not clear how well this technique would generalize to a real-world application, where a given file fragment is not known to belong to a file of only two possible types.

Axelsson \cite{Axelsson10} considered 28 different file types and applied the k-nearest-neighbors classification technique with nearest compression distance as the distance metric between file fragments. The file fragment data was generated from the freely available \texttt{govdocs1} corpus by Garfinkel et al. \cite{Garfinkel09}. Axelsson's experiments consisted of ten trials. In each trial, 10 files were selected at random (with the types of the files uniformly distributed) and 14 512-byte fragments were extracted at random from each of them. The fragments were then classified against a data set of approximately 3000 file fragments with known types. The average classification accuracy was around 34\%, with higher accuracy being achieved for file fragments with lower entropy.

Conti et al. \cite{Conti10} made use of a private data set of 14000 1024-byte binary fragments which they characterized by vectors of statistical measurements (namely, Shannon entropy, Hamming weight, Chi-square goodness-of-fit, and mean byte value). They classified each of these vectors against the remaining 13999 using k-nearest-neighbors with Euclidean distance as the distance metric. They achieved 98.55\% classification accuracy for Random/Compressed/Encrypted fragments, 100\% for Base64 Encoded fragments, 100\% for Unencoded fragments, 96.7\% for Machine Code (ELF and PE) fragments, 98.7\% for Text fragments, and 82.5\% for Bitmap fragments. However, the classifier did not perform as well when applied to real-world binary data, especially when it contained fragments of ``a previously unstudied primitive type, even one with a closely related structure''. They also classified the fragments according to types of a very coarse granularity.

Li et al. \cite{Li05} used the histogram of the byte values (unigram counts) of the prefix of a file (along with other portions of the file) in order to classify its type. They first collected a private data set of files across 8 different file types, and applied the k-means clustering algorithm to generate models for each file type (i.e. the centroids for the histograms of the files of that type). They achieved very good classification accuracy. However, this approach explicitly relies on the header information contained in each file, and hence, is not applicable to most file fragments which do not contain this information.

Veenman \cite{Veenman07} used the histogram of the byte values (unigram counts), the Shannon entropy, and the algorithmic or Kolmogorov complexity \cite{Kolmogorov65, Lempel76} as features for linear discriminant analysis to classify file fragments that were 4096 bytes in size. Veenman used a large private data set consisting of between 3000 and 20000 fragments per file type. Veenman achieved an average classification accuracy of 45\%.

Li et al. \cite{Li10} used a support vector machine with feature vectors based on the histogram of the byte values (unigram counts) to classify high entropy file fragments that were 4096 bytes in size. They used a private data set consisting of 880 \texttt{jpg} images, 880 \texttt{mp3} music files, 880 \texttt{pdf} documents, and 880 \texttt{dll} files for training and testing the support vector machine. They achieved an average classification accuracy of 81.5\%. It is likely that the large file fragment size made the classification task easier. However, in the absence of file system information in a real-world situation, a large file fragment size cannot be assumed. Furthermore, this work does not take into consideration other high entropy file types that might be of interest (such as \texttt{zip} or \texttt{gz} compressed files). Some differences between the work of Li et al. and ours are as follows. The data set we used is derived from a freely available corpus, making it easier to reproduce our work, unlike Li et al. We considered a file fragment size which can be safely assumed when no file system information is available \cite{Axelsson10}, unlike Li et al. Our classifier supports a much larger variety of file types, including both low and hight entropy ones, unlike Li et al.

\section{Experimental setup}
\label{Section:ExperimentalSetup}

\subsection{Data Set}
\label{Subsection:DataSet}

The data set we used for training and testing is derived from the freely available corpus of forensics research data by Garfinkel et al. \cite{Garfinkel09} (the \texttt{govdocs1} data set described in Section 4 of the cited paper). We determined the most well-represented file types in the data set and selected 23 of them based on how well-known they are to us. The first of these criteria ensured that we acquired a good variety of file fragment data for each file type. The second of these criteria is not a rigorous one. Although we aimed to get a good representation of file types that are likely to be of forensic interest, a rigorous methodology for selecting the most appropriate file types is outside the scope of this paper. Nevertheless, most of the file types we selected overlap with the ones selected by Axelsson \cite{Axelsson10} who made use of the same Garfinkel corpus to derive his data set.

After selecting the file types, we proceeded to download files uniformly at random from the \texttt{govdocs1} corpus such that we would have at least 10 files made up of at least 10000 512-byte fragments for each of the 23 file types. Because the files in the \texttt{govdocs1} corpus are of variable length, and files from different file types are not equally represented, it was necessary to download more than 10 files or files consisting of more than 10000 fragments, for each of the file types, in order to meet both criteria. From this, we uniformly at random selected 9000 fragments for each file type to create our data set. When generating the fragments, we omitted the first and last fragments of each file, as the first fragment frequently contains header information that identifies the file type, and the last fragment might not be 512 bytes in length. Calhoun and Coles \cite{Calhoun08}, Conti et al. \cite{Conti10}, and Li et al. \cite{Li10} omit these fragments as well. This approach enabled us to generate a large data set of file fragments with an equal number of file fragments for each file type, and with each fragment being derived from a variety of files with the same type.

\subsection{Feature Vectors}
\label{Subsection:FeatureVectors}
During the training phase, a support vector machine partitions a high-dimensional space based on data points with known classes, with each partition corresponding to a class. In order to train a support vector machine on the file fragment data, it is necessary to represent each file fragment as a point in a high-dimensional space. We do this by transforming each file fragment into a vector of features. The features we used are described here.

There are 256 features which are the histogram of the byte values (i.e. the unigram counts) for the file fragment. The feature vectors in the work by Li et al. \cite{Li10} consist only of the unigram counts. Another $256^2$ features in our work are the histogram of the pairs of consecutive byte values (i.e. the bigram counts) for the file fragment.

We also have features for various statistical measurements. We have the Shannon entropy \cite{Shannon48} of the bigram counts. Conti et al. \cite{Conti10} considered the Shannon entropy of the unigram, bigram, and trigram counts, and found that the entropy of the bigram counts was effective for classifying file fragments. We made use of the Hamming weight (the total number of ones divided by the total number of bits in the file fragment) and the mean byte value features that appear in the paper by Conti et al. We also used the compressed length of the file fragment as a feature. We did this to approximate the algorithmic or Kolmogorov complexity of the file fragment, following Veenman \cite{Veenman07}. We compressed each file fragment with the \texttt{bzip2} algorithm \cite{Seward01}. We also used two features from natural language processing. We computed the average contiguity between bytes (i.e. the average distance between consecutive bytes) for each file fragment, which is defined as follows: \\

$\sum_{i=0}^{n=510}\frac{|fragment[i] - fragment[i+1]|}{511}$ \\

{\noindent}where $fragment[i]$ is the $i^\textrm{\small th}$ byte of the file fragment. Last of all, we calculated the longest contiguous streak of repeating bytes for each file fragment.

\subsection{Experiments}
\label{Subsection:Experiments}

%After finalizing our data set, we split up our experiment based on how many fragments of each file type will be used to test and train the SVM. This is done in order for us to be able to asses the value of using more fragments of a given file type versus using less fragments. We divided our experiment into 1000 fragments of each type, 2000, 4000, and finally 8000; obviously, the SVM training phase time increases as the amount of fragments being employed increases. If the prediction rate of using less fragments versus using considerably more fragments is negligible, then in most cases, it will suffice to use less fragments for the SVM training phase, if time is a concern.





\section{Results}
\label{Section:Results}
\textbf{TODO finish this!!!!!!}

\section{Conclusion and future work}
\label{Section:ConclusionAndFutureWork}
File fragment classification is an important problem in digital forensics. For this paper, we explored the application of supervised machine learning techniques from natural language processing to this problem. We generated a large data set of file fragments for 23 different file types, which we derived from the freely available \texttt{govdocs1} corpus by Garfinkel et al. \cite{Garfinkel09}. We represented each file fragment with a feature vector consisting of the unigram and bigram counts of bytes in the fragment, along with several other statistical measurements (such as entropy). Our file fragment classification approach consisted of using a support vector machine along with these feature vectors. We ran several experiments, and found this to be an effective and robust approach.

\textbf{TODO when experiments are done!!}

\section*{Acknowledgements}
We are grateful to G. Scott Graham for suggesting the initial direction for this project, and to Greg Conti for his help.


% References
\newpage
\bibliographystyle{plain}
\bibliography{report}

\end{document}
